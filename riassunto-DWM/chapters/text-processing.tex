\chapter{Text Processing}
    Di seguito alcuni metodi per processare dei testi:

    \begin{itemize}
        \item $\textbf{TF-IDF}$ (Term Frequency - Inverse Document Frequency): è un algoritmo che utilizza la frequenza delle parole per determinare quanto tali parole siano rilevanti per un dato documento.
        \\
        È un approccio relativamente semplice ma intuitivo per dare peso alle parole.
        \\[1\baselineskip]
        Per riassumere l'intuizione chiave, l'importanza di un termine è inversamente correlata alla sua frequenza nei documenti.
        \\[1\baselineskip]
        TF ci fornisce informazioni sulla frequenza con cui un termine $t$ appare in un documento $d$ e IDF ci fornisce informazioni sulla relativa rarità di un termine $t$ nella raccolta dei documenti $D$.
        Moltiplicando questi valori insieme possiamo ottenere il nostro valore TF-IDF finale.

            $$ tf\ idf(t, d, D) = tf(t, d) \cdot idf(t, D) $$

        dove $idf(t, D)$ è definito come:
            $$ idf(t, D) = \ln \left( \frac{N_{\textrm{docs}}}{df(t)} \right)$$
        
        in cui $N_{\textrm{docs}}$ rappresenta il numero di documenti nella collezione $D$ mentre $df(t)$ è il numero di documenti contenenti il termine $t$;
        \\[0.5\baselineskip]

        \item $\textbf{Stemming}:$ viene usato per rimuovere il prefisso/suffisso (es: "being" $\longrightarrow$ "be", "was" $\longrightarrow$ "was");
            \\[0.5\baselineskip]
        \item $\textbf{Lemming}:$ si riferisce all'identificazione dell'origine della parola (es: "being" $\longrightarrow$ "be", "was" $\longrightarrow$ "be").
    \end{itemize}

    \clearpage