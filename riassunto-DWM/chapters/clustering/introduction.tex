\section{Introduzione}
    $\textbf{L'analisi dei Clusters}$ si pone l'obiettivo di trovare le similarità tra i dati secondo le caratteristiche trovate nei dati stessi e raggruppare i dati simili tra loro in $\textbf{clusters}$, ovvero delle collezioni di dati.
    \\[1\baselineskip]
    Questo tipo di problemi fanno parte della seconda famiglia di apprendimento, ovvero $\textbf{Unsupervised Learning}$: non ci sono etichette di classe già predefinite, ma l'algoritmo deve imparare da solo attraverso l'osservazione e l'analisi dei dati.
    \\[1\baselineskip]
    Il risultato di un'ottima procedura di clustering prevede come risultato dei clusters che hanno:
        \begin{itemize}
            \item $\textbf{Alta Similarità "Intra-Class"}:$ ogni dato è molto simile a tutti gli altri dati del cluster;
            \item $\textbf{Bassa Similarità "Inter-Classe"}:$ ogni dato è molto dissimile da tutti gli altri dati fuori dal cluster.
                \\[1\baselineskip]
        \end{itemize}

    Tra i vari approcci per la risoluzione dei problemi di clustering, i tre più comuni sono:
        \begin{itemize}
            \item $\textbf{Partitioning}:$ vengono costruite varie partizioni e valutate secondo alcuni criteri (es: minimizzazione del Sum of Square Errors);
            \item $\textbf{Hierarchical}:$ viene creata una decomposizione gerarchica del dataset usando alcuni criteri;
            \item $\textbf{Density-Based}:$ approccio basato su funzioni di densità e connettività.
        \end{itemize}

\clearpage